\documentclass{scrartcl}

\deffootnote{1em}{1em}{\textsuperscript{\thefootnotemark}}

\addtokomafont{disposition}{\rmfamily}
\addtokomafont{descriptionlabel}{\rmfamily}

\RequirePackage
    [bitstream-charter,greekuppercase=italicized]
    {mathdesign}
\renewcommand{\sfdefault}{fvs}
\renewcommand{\ttdefault}{fvm}

\usepackage[utf8x]{inputenc}

\usepackage{microtype}

\usepackage[hidelinks]{hyperref}

\usepackage{amsmath}

\newenvironment{mathfigure}[2]
    {%
        \begin{figure}
        \newcommand{\figurelabel}{#1}
        \newcommand{\figurecaption}{#2}
        \centering
        \begin{math}
    }
    {
        \end{math}
        \caption{\figurecaption}
        \label{\figurelabel}
        \end{figure}%
    }

\newcommand{\bnfdef}{\mathrel{{\mathop:}{\mathop:}{=}}}

\newcommand{\optional}{^?}
\newcommand{\some}{^+}
\newcommand{\many}{^*}

\newcommand{\newnt}[1]
           {\expandafter\newcommand\csname#1\endcsname{\mathit{#1}}}

\newnt{Char}
\newnt{Letter}
\newnt{Digit}

\newnt{Ident}
\newnt{Str}
\newnt{Nat}
\newnt{Keyword}
\newnt{Symbol}

\newnt{Prog}
\newnt{Assign}
\newnt{Expr}
\newnt{Let}
\newnt{If}
\newnt{Lam}
\newnt{CLam}
\newnt{App}
\newnt{List}
\newnt{TNat}
\newnt{FNat}
\newnt{Pair}
\newnt{Left}
\newnt{Right}
\newnt{Trace}
\newnt{Var}

\newcommand{\newkw}[1]
           {\expandafter\newcommand\csname kw#1\endcsname{\mathbf{#1}}}

\newkw{let}
\newkw{in}
\newkw{if}
\newkw{then}
\newkw{else}
\newkw{left}
\newkw{right}
\newkw{trace}

\title{Specification of the Stand-in Language}
\author{%
    Wolfgang Jeltsch\\
    \small Well-Typed LLP\\
    \small\texttt{wolfgang@well-typed.com}%
}

\begin{document}

\maketitle

\section{Syntax}

We describe the syntax of SIL using a variant of Backus–Naur form. Our
variant uses the following notations, listed here in increasing order of
precedence:
\begin{itemize}

\item

Words in italics are nonterminals.

\item

$∣$ denotes set union.

\item

$∖$ denotes set difference.

\item

Mere juxtaposition denotes concatenation.

\item

$\optional$, $\some$, and $\many$ denote optionality, repetition with at
least one occurrence, and arbitrary repetition, respectively.

\item

$⟨$~and~$⟩$ delimit subexpressions and are used for overriding default
precedence.\footnote{We do not use $($ and~$)$ for this purpose, since
they are part of the language we want to describe.}

\end{itemize}

Figure~\ref{token-syntax-of-sil} defines the syntax of tokens. Based on
that, Figure~\ref{syntax-of-sil} defines the syntax of the language. The
nonterminals $\CLam$, $\TNat$, and $\FNat$ stand for “complete lambda”,
“tuple natural”, and “function natural”, respectively.
\begin{mathfigure}{token-syntax-of-sil}{Token syntax of SIL}
%
\begin{aligned}
%
\Ident   & \bnfdef \Letter⟨\Letter ∣ \Digit ∣ \_ ∣ {′}⟩\many ∖
                   \Keyword                                          \\
\Str     & \bnfdef \text{“}\Char\many\text{”}                        \\
\Nat     & \bnfdef \Digit\some                                       \\
\Keyword & \bnfdef \kwlet ∣ \kwin ∣ \kwif ∣ \kwthen ∣ \kwelse ∣
                   \kwleft ∣ \kwright ∣ \kwtrace                     \\
\Symbol  & \bnfdef {(} ∣ {)} ∣ {[} ∣ {]} ∣ {\{} ∣ {\}} ∣ {:} ∣ {=} ∣
                   \backslash ∣ \# ∣ {→} ∣ {,} ∣ \$
%
\end{aligned}
%
\end{mathfigure}
\begin{mathfigure}{syntax-of-sil}{Syntax of SIL}
%
\begin{aligned}
%
\Prog   & \bnfdef \Assign\some                                       \\
\Assign & \bnfdef \Ident \; ⟨{:} \; \Expr₀⟩\optional = \Expr₀        \\
\Expr₀  & \bnfdef \Let ∣ \If ∣ \Lam ∣ \CLam ∣ \Expr₁                 \\
\Let    & \bnfdef \kwlet \; \Assign\many \; \kwin \; \Expr₀          \\
\If     & \bnfdef \kwif \; \Expr₀ \; \kwthen \; \Expr₀ \; \kwelse \;
                  \Expr₀                                             \\
\Lam    & \bnfdef \backslash \; \Var\some → \Expr₀                   \\
\CLam   & \bnfdef \# \; \Var\some \to \Expr₀                         \\
\Expr₁  & \bnfdef \App ∣ \Expr₂                                      \\
\App    & \bnfdef \Expr₁ \; \Expr₂                                   \\
\Expr₂  & \bnfdef \List ∣ \Str ∣ \TNat ∣ \FNat ∣ \Pair ∣ \Left ∣
                  \Right ∣ \Trace ∣ \Var ∣ (\Expr₀)                  \\
\List   & \bnfdef [] ∣ [\Expr₀ ⟨, \Expr₀⟩\many]                      \\
\TNat   & \bnfdef \Nat                                               \\
\FNat   & \bnfdef \$\Nat                                             \\
\Pair   & \bnfdef \{\Expr₀, \Expr₀\}                                 \\
\Left   & \bnfdef \kwleft \; \Expr₂                                  \\
\Right  & \bnfdef \kwright \; \Expr₂                                 \\
\Trace  & \bnfdef \kwtrace \; \Expr₂                                 \\
\Var    & \bnfdef \Ident                                             \\
%
\end{aligned}
%
\end{mathfigure}
%NOTE: The vertical bars in the grammar source are not the ASCII
%      character U+007C VERTICAL LINE but the character U+2223 DIVIDES,
%      which the ucs package translates into \mid. This is important to
%      get the proper spacing.

\end{document}
